\section{\label{sec:intro}Introduction}
Underlying traditional mean-field mass-action models that are applied in the study of population dynamics as well as chemical reaction networks is a first-order Markovian stochastic process on a state space comprising the positive orthant of a hypercubic lattice. This is because in models of population dynamics as well as chemical reaction networks the traditional state variables represent the fraction of a given type of entity, whether they be biological organismic or chemical species. Because it is more difficult to analyze the underlying stochastic process, whether implicitly or explicitly, the assumptions necessary to justify the mean-field ODE models that dominate literature on both population dynamics and chemical reaction networks are imposed.


Using methods from quantum field theory at phenomenological levels above their domain of initial development, we demonstrate how the stochastic process underlying the traditional mean-field models can be solved in some generality for reaction networks applicable to both population dynamic and chemical reaction network models. Modeling such systems at this level, enables one to take into account fluctuations which are necessarily neglected by mean-field models, while the methods we apply are nearly as general as those used to study mean-field models. We hope that combined, the latter two facts may help to push the boundary in the understanding of the applicability and development of models of reaction networks, regardless of the particular level of organization at which they are applied.
