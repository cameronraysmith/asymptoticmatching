\section{Results}\label{sec:results}

\subsection{The master equation for reaction networks}
The general form of the master equation for reaction networks in the conventional basis of state probabilities is
% \begin{widetext}
\begin{align*}
\frac{d P(n_i,t)}{dt} = \sum_{\mu \rightarrow \mu' \colon R} k_{\mu \rightarrow \mu'} \left(\prod_{i \colon S} \binom{n_i + \mu_i - \mu'_i}{\mu_i} P(n_i + \mu_i - \mu'_i,t) - \prod_{i \colon S} \binom{n_i}{\mu_i} P(n_i,t) \right)
\end{align*}
% \end{widetext}
where $n_i$ represents a state vector over species $S$, $R$ is the set of reactions between complexes $\mu, \mu', \ldots$, and $k_{\mu \rightarrow \mu'}$ is the rate constant for the reaction taking complex $\mu$ into $\mu'$.

For the purpose of computation, it often turns out to be easier to
manipulate a generating function rather than compute with the
probabilities directly.  In order to introduce generating functions, we
shall make use of a technique due to Doi and Pellitti.  The starting
point of their construction is the algebra of ladder operators for the
harmonic oscillator.  In the choice of normalization which we shall use
in this article, this algebra and its representation by occupation
number states is specified as follows:
\begin{align*}
a_i a_j &= a_j a_i \cr
a^\dagger_i a^\dagger_j &= a^\dagger_j a^\dagger_i \cr
a_i a^\dagger_j &= a^\dagger_j a_i + \delta_{ij} I \cr
a^\dagger_i |n_1, \ldots, n_N \rangle &=
|n_1, \ldots, n_i + 1, \ldots n_N \rangle \cr
a_i |n_1, \ldots, n_N \rangle &=
n_i |n_1, \ldots, n_i - 1, \ldots n_N \rangle \cr
\langle n_1, \ldots, n_N | a_i &=
\langle n_1, \ldots, n_i + 1, \ldots n_N | n_i +1 \cr
\langle n_1, \ldots, n_N | a^\dagger_i &=
\langle n_1, \ldots, n_i + 1, \ldots n_N | \cr
\langle m_1, \ldots, m_N | n_1, \ldots n_N \rangle &=
\delta_{m_1,n_1} \cdots \delta_{m_N n_N}
\end{align*}
We will also make use of coherent states, which are defined by the
following formulae:
\begin{align*}
\Bra{\alpha} &= \Bra{0} e^{\alpha^* a} \\
\Ket{\alpha} &= e^{\alpha a^{\dagger}} \Ket{0}\\
\Bra{\alpha}a^\dagger &= \Bra{\alpha}\alpha^*\\
a \Ket{\alpha} &= \alpha \Ket{\alpha}\\
\Braket{\alpha|\beta} &= e^{\alpha^* \beta}.
\end{align*}
Because
$$
\frac{d^n}{d \alpha^n} \Ket{\alpha} = a^{\dagger n} \Ket{\alpha},
$$
the relationship of the coherent states to the $\Ket{n}$ state is given by
$$
\Ket{n} = \left. \frac{d^n}{d \alpha^n} \Ket{\alpha} \right|_{\alpha=0}.
$$
the reference state with respect to which all others are normalized is the coherent state
$$\Bra{\alpha \mathcal{S}} = \bra{0}e^{\alpha a}$$
where $\alpha\equiv1$. The norm of a given normalized state $\ket{n}=a^{\dagger n} \ket{0}$ is then
$$\Braket{\mathcal{S}|n}=1$$

We define the generating function as follows:
$$
| \psi (t) \rangle =
\sum_{n_1 = 1}^\infty \cdots \sum_{n_N = 1}^\infty
P (n_1, \ldots, n_N, t) |n_1, \ldots, n_N \rangle
$$
Upon substituting this into the master equation and simplifying, we find
that this quantity satisfies differential equation ${\partial \psi \over
\partial t} = - H \psi$ where the operator $H$ is defined as
\begin{align*}
H = \sum_{\mu \rightarrow \mu' \colon R} \frac{k_{\mu \rightarrow \mu'}}{\prod_i \mu_i !} \left(\prod_{i \colon S} {a_i^{\dagger}}^{\mu_i}{a_i^{}}^{\mu_i} - \prod_{i \colon S} {a_i^{\dagger}}^{\mu'_i}{a_i^{}}^{\mu_i} \right)
\end{align*}
In terms of these quantities, the conditional probabilities are expressed as
\begin{align*}
&P(m_1, \ldots, m_N; t_2 | n_1,\ldots,n_N ; t_1)\\
&= \Braket{m_1, \ldots, m_N| e^{-H(t_2-t_1)}|n_1, \ldots, n_N}.
\end{align*}
The number operator $\hat{n} \equiv a^{\dagger} a$ extracts the coefficient from $n$ and we can thus compute the expectation value as
\begin{align*}
\bra{\mathcal{S}}\hat{n} \ket{\psi(t)} &= \bra{\mathcal{S}}a^{\dagger} a \ket{\psi(t)}\\
&= \sum_{n} n P(n,t) \\
&= \braket{n}
\end{align*}
which generalizes to compute the expectation of any operator-valued function of the number operator
$$\bra{\mathcal{S}} f(a^{\dagger} a) \ket{\psi(t)} = \sum_{n} f(n) P(n,t)=\braket{f}(t)$$

Having presented this technique, we would like to forestall any possible
confusion by pointing out that what we are doing here is to make use of
the mathematical apparatus and notation of quantum mechanics to describe
a purely classical system --- our probabilities are computed by summing
partial probabilities rather squares of amplitudes; our operators are
stochastic matrices rather Hermitean matrices and our exponentials are
real rather than complex.  Nevertheless, it turns out that when we write
things this way, the resulting expressions bear a formal similarity to
counterparts from quantum theory and that this resemblance successfully
suggests techniques which are of use in analyzing our problem.

\subsection{Interaction term expansion}
For the purpose of analysis, $H$ can be decomposed into terms $H_0$, representing relatively low probabiltiy fluctuations between species, and $H_1$, an interaction term representing relatively higher probability reactions. In this decomposition the master equation takes the form
\begin{align*}
\frac{d \psi}{dt} = -(H_0+cH_1)\psi
\end{align*}
and has solutions
\begin{align*}
\psi(t) = \exp(-(H_0+cH_1)t)\psi(0).
\end{align*}
Choosing a specific form for $H_0$ that is bilinear in ladder operators and conserves total particle number
\begin{align*}
H_0 &= \sum_{ij} a^{\dagger i} {h_{i}}^{j} a_{j}
\end{align*}
we can write the unequal time commutation relations for the ladder operators in terms of ${h_{i}}^{j}$ as
\begin{align*}
\left[ a_{i}(t_1),a^{\dagger j}(t_2) \right] &= {\left(e^{{h_{*}}^{*} (t_1 - t_2)}\right)_{i}}^{j}.
\end{align*}
We also have time-dependent states
\begin{align*}
\Bra{n_1,\ldots,n_N,t} &= \Bra{n_1,\ldots,n_N} e^{H_0 t} \cr
&= \Bra{0} (a_1 (t)^{n_1}) \cdots (a_N (t)^{n_N})\\
\Ket{n_1,\ldots,n_N,t} &= e^{-H_0 t} \Ket{n_1,\ldots,n_N} \cr
&= (a^{\dagger}_1 (t)^{n_1}) \cdots (a^{\dagger}_N (t)^{n_N})\Ket{0}
\end{align*}
upon which the time-dependent ladder operators act as
\begin{align*}
a_i(t_1) &\Ket{n_1,\ldots,n_N,t} = \cr
&\sum_{j=1}^N n_j {{e^{h(t_1-t)}}_{i}}^{j} \Ket{n_1,\ldots,n_j-1,n_N,t},\\
\Bra{n_1,\ldots,n_N,t}& a^{\dagger}_i(t_1)  = \cr
&\sum_{j=1}^N \Bra{n_1,\ldots,n_j+1,n_N,t} {{e^{h(t_1-t)}}_{i}}^{j}.
\end{align*}
We can then reexpress $H_1$ and the time-dependent ladder operators using similarity transformations
\begin{align*}
H_1(t) &= e^{-H_0 t} H_1 e^{H_0 t},\\
{a_{\mu}}^{\dagger}(t) &= e^{-H_0 t} {a_{\mu}}^{\dagger} e^{H_0 t},\\
{a_{\mu}}(t) &= e^{-H_0 t} {a_{\mu}} e^{H_0 t}.
\end{align*}
allowing us to write solutions in terms of the time ordering operator $\mathcal{T}$
\begin{align*}
\exp &(-(H_0 + cH_1)t) \cr
&= \mathcal{T} \Biggl[ \exp \Biggl(\int_0^t dt_1 (-c H_1(t_1))\Biggr) \Biggr] \exp(-H_0 t).
\end{align*}
Using Wick's theorem, we can convert the time-ordering to normal-ordering

\subsection{Appearance of new species and scaling}

Suppose that we have two species, $A$ and $B$ and the reaction $A \to B$ with rate constant $k$.  Then $H = k (-b^\dagger a + a^\dagger a)$ and the probability of a single $B$ being observed at time $t$ if we start with zero $B$'s at time $t=0$ is given by
\[
 \braket{1, N-1 | e^{-Ht} | 0, N}
\]
We can expand $e^{-Ht}$ as a series in powers of $b^\dagger a$:
\[
 e^{-Ht} = e^{-k a^\dagger a} +
   k \int_0^t dt'\, e^{-k a^\dagger a t'} (b^\dagger a) e^{-k a^\dagger a (t - t')}
   + \cdots
\]
Only the second term will contribute to the transition probability.  Furthermore, the operator $a^\dagger a$ in the exponents can be replaced by the number of $A$'s to give
\begin{align*}
 &\!\braket{1, N-1 | e^{-Ht} | 0, N} \cr &=
 k \int_0^t dt'\,\braket{1, N-1 | e^{-k a^\dagger a t'} (b^\dagger a) e^{-k a^\dagger a (t - t')} | 0, N} \cr &=
 k \int_0^t dt'\, e^{-k (N-1) t' - k N (t - t')} \braket{1, N-1 | b^\dagger a | 0, N} \cr &=
 kN \int_0^t dt'\, e^{kt' - kNt} = N (e^{kt} - 1) e^{-kNt}
\end{align*}
In order to examine scaling with $N$, let us compute the time at which this quantity attains its maximum:
\begin{align*}
 0 &= \frac{d}{dt} \braket{1, N-1 | e^{-Ht} | 0, N} \cr &=
 N \left( k (1-N) e^{k (1-N) t} + kN e^{-kNt} \right)
 \end{align*}
 Solving this equation, we obtain $t_\mathrm{max} = \frac{1}{k} \log \frac{N}{N-1}$.  For $N$ large, we have $\log(N/N-1) \approx 1/N$, so we conclude that $k$ scales as $1/N$.  Hence, let us set $k = 1/TN$ where $T$ is a constant with dimensions of time.  Substituting into our transition probabilty, we obtain
 \begin{align*}
 \braket{1, N-1 | e^{-Ht} | 0, N} =
 N \left( e^\frac{t}{NT} - 1 \right) e^{-\frac{t}{T}}
 \end{align*}
 Note that, $\lim_{N \to \infty} N \left( e^\frac{t}{NT} - 1 \right) = \frac{t}{T}$  so our quantity is well-behaved in the limit and, for large $N$, we have $\braket{1, N-1 | e^{-Ht} | 0, N} \approx \frac{t}{T} e^{-\frac{t}{T}}$.

Next, we will generalize this calculation to the case where we add the reverse reaction $B \to A$.  Letting $k_1$ be the rate constant for the reaction $A \to B$ and $k_2$ be the constant for $B \to A$, we have
\[
 H = \begin{pmatrix} b^\dagger & a^\dagger \end{pmatrix} h \begin{pmatrix} b \cr a \end{pmatrix}
\]
where
\[
 h = \begin{pmatrix} k_2 & -k_1 \cr -k_2 & k_1 \end{pmatrix}
\]
Note that
\begin{align*}
 e^{-H_0 t} \ket{0, N} &=
 e^{-H_0 t} (a^\dagger)^N \ket{0, 0} \cr &=
 e^{-H_0 t} (a^\dagger)^N e^{H_0 t} \ket{0, 0} \cr &=
 (a^\dagger (t))^N \ket{0, 0}
\end{align*}
where, in the second line of the derivation, we use the fact that $e^{Ht} \ket{0, 0} = \ket{0, 0}$ which follows from expanding the exponential and noting that $H \ket{0, 0} = 0$.  We can obtain $a^\dagger (t)$ from the formula
\[
 \begin{pmatrix} b^\dagger(t) & a^\dagger(t) \end{pmatrix} =
 \begin{pmatrix} b^\dagger & a^\dagger \end{pmatrix} e^{-ht} .
\]
In order to compute the exponential of $ht$, we start with the observation that $h^2 = (k_1 + k_2) h$ which can readily be verified from the definition of $h$.  Iterating, we have $h^{n+1} = (k_1 + k_2)^n h$.  Thus, if we expand the exponential, we can apply this formula and resum the terms to obtain
\[
 e^{-ht} = I + \frac{e^{-(k_1 + k_2) t} - 1}{k_1 + k_2} h
\]
where $I$ is the $2 \times 2$ identity matrix.  Thus,
\[
 a^\dagger (t) = a^\dagger +
 \frac{e^{-(k_1 + k_2) t} - 1}{k_1 + k_2} k_1 (a^\dagger - b^\dagger) ,
\]
whence
\begin{align*}
 &\!\braket{1, N-1 | e^{-Ht} | 0, N} \cr &=
 \bra{1, N-1} \left( a^\dagger (t) \right)^N \ket{0, 0} \cr &=
 -N \bra{1, N-1} \frac{e^{-(k_1 + k_2) t} - 1}{k_1 + k_2} k_1 b^\dagger \times \cr
   &\hskip2cm \left( \frac{k_1 e^{-(k_1 + k_2) t} + k_2}{k_1 + k_2} a^\dagger \right)^{N-1} \ket{0, 0} \cr &=
 N \frac{1 - e^{-(k_1 + k_2) t}}{k_1 + k_2} k_1 \left( \frac{k_1 e^{-(k_1 + k_2) t} + k_2}{k_1 + k_2} \right)^{N-1}
\end{align*}
Note that when $k_2 = 0$, this expression reduces to $N(1-e^{-k_1 t})(e^{-k_1 t})^{N-1} =N (e^{k_1 t} - 1) e^{-N k_1 t}$, which is exactly the expression we obtained previously.

There are two different ways of rescaling the rate constants so as to obtain a well-defined limit as $N$ becomes large.

Firstly, we can scale both couplings as $1/N$:
\begin{align*}
k_1 &= \frac{c_1}{N} \\
k_2 &= \frac{c_2}{N} \\
\end{align*}
With this scaling, our transition probability becomes
\begin{align*}
 &\!\braket{1, N-1 | e^{-Ht} | 0, N} \cr &=
 \frac{c_1}{c_1 + c_2} N \left( 1 - e^{-\frac{c_1 + c_2}{N} t} \right)
 \left( 1 + \frac{c_1}{c_1 + c_2} \left( e^{-\frac{c_1 + c_2}{N} t} - 1 \right) \right)^{N-1}
\end{align*}
This has a well defined limit:
\begin{align*}
 \lim_{N \to \infty} \braket{1, N-1 | e^{-Ht} | 0, N} =
 c_1 t e^{-c_1 t}
\end{align*}

Secondly, we can scale $k_1$ by $1/N$  but leave $k_2$ constant.
\begin{align*}
k_1 &= \frac{c_1}{N} \\
k_2 &= c_2 \\
\end{align*}
With this scaling, our transition probability becomes
\begin{align*}
 &\!\braket{1, N-1 | e^{-Ht} | 0, N} \cr &=
 \frac{c_1}{\frac{c_1}{N} + c_2}
 \left( 1 - e^{-(\frac{c_1}{N} + c_2) t} \right)
 \left( \frac{1 + \frac{c_1}{c_2 N} e^{-(\frac{c_1}{N} + c_2) t}}{1 + \frac{c_1}{c_2 N}} \right)^{N-1}
\end{align*}
This has a well defined limit:
\begin{align*}
 \lim_{N \to \infty} &\braket{1, N-1 | e^{-Ht} | 0, N} \\ &=
 \frac{c_1}{c_2}
 \left( 1 - e^{-c_2 t} \right)
 e^{- \frac{c_1}{c_2} \left( 1 - e^{-c_2 t} \right)}
\end{align*}

Next, we generalize this to situations in which we have more than one transition.  This will be needed for the perturbation calculations.
\begin{align*}
\braket{ m_1, m_2 | e^{-H_0 t} | n_1 n_2 } &= 
\braket{ m_1, m_2 | (h_{11} {\tilde a}_1^\dagger +
                     h_{12} {\tilde a}_2^\dagger)^{n_1}
                    (h_{21} {\tilde a}_1^\dagger +
                     h_{22} {\tilde a}_2^\dagger)^{n_2}
 | 0, 0} \\ &\qquad=
\sum_{k_1 = 0}^{n_1} \sum_{k_2 = 0}^{n_2}
 {n_1 \choose k_1} {n_2 \choose k_2}
 h_{11}^{k_1} h_{12}^{n_1 - k_1}
 h_{21}^{k_2} h_{22}^{n_2 - k_2}
 \braket{ m_1, m_2 |
  ({\tilde a}_1^{\dagger})^{k_1 + k_2}
  ({\tilde a}_2^\dagger)^{N - k_1 - k_2} | 0, 0 } \\ &\qquad=
\sum_{k_1 = 0}^{n_1} \sum_{k_2 = 0}^{n_2}
 {n_1 \choose k_1} {n_2 \choose k_2}
 h_{11}^{k_1} h_{12}^{n_1 - k_1}
 h_{21}^{k_2} h_{22}^{n_2 - k_2}
 \braket{ m_1, m_2 | k_1 + k_2, N - k_1 - k_2 } \\ &\qquad=
\sum_{k_1 = 0}^{n_1} \sum_{k_2 = 0}^{n_2}
 {n_1 \choose k_1} {n_2 \choose k_2}
 h_{11}^{k_1} h_{12}^{n_1 - k_1}
 h_{21}^{k_2} h_{22}^{n_2 - k_2}
 \delta_{m_1 \, k_1 + k_2} \delta_{n_2 \, N - k_1 - k_2} \\
&\qquad= \sum_{k_1 = \max (0, m_1 + n_1 - N)}^{\max (m_1, n_1)}
 {n_1 \choose k_1} {N - n_1 \choose m_1 - k_1}
 h_{11}^{k_1} h_{12}^{n_1 - k_1}
 h_{21}^{m_1 - k_1} h_{22}^{N - m_1 - n_1 + k_1}
\end{align*}



\subsection{System size $\frac{1}{N}$ expansion}
